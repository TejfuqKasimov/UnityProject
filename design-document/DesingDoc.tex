\documentclass[a4paper,12pt]{article}
\usepackage{graphicx}
\usepackage{hyperref}
\usepackage{geometry}
\usepackage[utf8]{inputenc}  % Для кодировки UTF-8
\usepackage[russian]{babel}   % Для поддержки русского языка
\geometry{top=1in, bottom=1in, left=1in, right=1in}
\usepackage[margin=1in]{geometry}
\usepackage{titlesec}
\usepackage{graphicx} % Для работы с изображениями
\usepackage{float}    % Для контроля размещения изображений
\titleformat{\section}{\large\bfseries}{\thesection.}{1em}{}
\titleformat{\subsection}{\normalsize\bfseries}{\thesubsection.}{1em}{}


\title{Дизайн-документ 2D Платформера}
\date{\today}

\begin{document}

\maketitle

\section{Введение}
Данный документ представляет собой дизайн-документ для 2D платформера. Его цель — предоставить полное описание концепции, игровых механик, технических особенностей, графического и звукового оформления, а также других аспектов разработки. Для эффективного использования данного документа ознакомьтесь с нижеприведенной информацией.

\subsection{Организация содержимого документа}
Документ разделен на тематические разделы, охватывающие ключевые аспекты разработки игры. Основные главы включают:
\begin{itemize}
    \item Введение
    \item Концепция
    \item Функциональная спецификация
\end{itemize}
Каждый раздел имеет чёткую структуру и содержит ссылки на связанные разделы для удобства навигации.

\subsection{Ссылки и используемые материалы}
Для подготовки документа были использованы следующие материалы:
\begin{itemize}
    \item Открытые библиотеки игровых шрифтов, ассетов и референсов, соответствующих лицензиям Creative Commons.
    \item Документация используемых движков и фреймворков, включая \textit{Unity}.
\end{itemize}
Все материалы защищены соответствующими авторскими правами и используются в рамках установленных лицензий.
\subsection{История изменений}
\begin{itemize}
    \item Версия 1.0 (17.11.2024) — начальная версия документа, созданы главы: Введение, Концепция
\end{itemize}

\subsection{Список авторов}
Документ подготовлен следующими авторами:
\begin{itemize}
    \item Орешкин Максим
    \item Выставкин Константин
    \item Касимов Тейфук
    \item Царапкин Егор
    \item Яновский Дмитрий
    \item Смыков Виктор
\end{itemize}

\subsection{Условные обозначения, сокращения и соглашения}
\begin{itemize}
    \item \textbf{2D} — двумерный формат графики.
\end{itemize}

\subsection{Прочие сведения}
Для чтения и понимания документа желательно иметь общее представление о разработке игр, включая основные концепции геймдизайна, программирования и художественного оформления. При необходимости, разделы документа содержат пояснения для менее знакомых терминов и понятий.

\section{Концепция}
\subsection{Введение}
В 2D-шутере игрок управляет рыцарем с загадочным именем «Лидарь», уворачиваясь от летящих дисков бензопилы и беспощадной турели «Чаппи», плюющейся своим огненным дождём из пуль, чтобы достичь портала и перейти на следующую локацию.
Динамичный геймплей рассчитаны на любителей быстрых реакций, поэтому стоит думать и, конечно же, не стоит медлить, мой друг. Строго 6+.

\subsection{Жанр и аудитория}
Сведения о жанре и целевой аудитории:

• Жанр
    
    2D-шутер

• Возрастная группа
    
    Дети 6+, обладающие наименьшей степенью реакций и взрослые: любители аркад и шутеров (от третьего лица), а также игроки, которые хотят прокачать быстроту реакции.

• Другие сведения о позиционировании игры
    
    Благодаря простоте управления и одновременно высокой сложности прохождения последних уровней игра понравится как любителям, так и новичкам.

\subsection{Основные особенности игры}
\begin{itemize}
    \item Механика игры отличается своей простотой.
    \item Особенность подобранных фонов в игре может порадовать игрока при прохождении.
    \item При прохождении игроку нужна бдительность своих действий и ловкость рук, что делает её весьма интересной.
    \item Необычное сочетание вражеского оружия (диски бензопилы и пули).
    \item Перемещение между уровнями оформлено в виде телепорта.
    \item Постепенное увеличение сложности.
    \item Среднее прохождение игры 12-17 минут
\end{itemize}

\subsection{Описание игры}
Игрок управляет рыцарем Лидарём, который должен пройти серию из 5-ти уровней, избегая попадания вражеских снарядов по нему. Враги атакуют Лидаря с помощью вращающихся кровавых дисков бензопилы и пуль, испускаемых турелями.
Главная цель игрока на каждом уровне — добраться до портала, чтобы перейти на следующий. Сложность игры возрастает с каждым новым уровнем, увеличивая количество и скорость вражеских снарядов.

\subsection{Предпосылки создания}
Новаторский 2D-шутер с использованием градиентной стилизованной графики поможет расширить спрос на рынке аркадных игр.
Простая механика и несложный геймплей, но важна заинтересованность и вовлечённость пользователя - он вырабатывается из-за нарастающей сложности игры. 
Лицензирование не требуется.

\subsection{Платформа}
Перечислите платформы, на которых планируется создание игры. Для РС-платформы укажите минимальные и рекомендуемые системные требования. Если игра требует дополнительного оборудования (например, модем), укажите это.  
\begin{table}[h!]
\centering
\begin{tabular}{|c|c|c|}
\hline

Требования & \textbf{Минимальные} & \textbf{Рекомендуемые} \\ \hline
Операционная система & Windows 7 & Windows 10\\ \hline
Процессор& 2 ядра, 2.0 ГГц & 4 ядра, 3.0 ГГц\\ \hline
ОЗУ& 4 ГБ& 8 ГБ\\ \hline
CD-ROM привод & - & - \\ \hline
Свободное место на HDD&200 МБ & 1ГБ\\ \hline
Видеокарта & Графическое ядро & 1 ГБ видеопамяти \\ \hline
Звуковая карта& - & - \\ \hline
Управление& Клавиатура, мышь & Клавиатура, мышь\\ \hline
\end{tabular}

\caption{Системные требования}
\label{tab:example_table}
\end{table}


\section{Функциональная спецификация}
Функциональная спецификация — основное описание игры с точки зрения игрока. Именно здесь раскрываются все возможности игры. Избегайте технических деталей или выкладок с точки зрения разработчика. Содержание данного раздела рассматривается как предстоящий объем разработки — по умолчанию принимается, что все возможности, изложенные здесь, будут реализованы, а то, что не упоминается, нет. Объем может очень существенно отличаться в зависимости от жанра, качества проработки и т.д., но вместить изложение всех игровых вопросов в 20 страниц вряд ли реально.

\subsection{Принципы игры}


\subsubsection{Суть игрового процесса}
"The game" представляет из себя многофункциональный платформер, где игрому предстоит столкнуться со множеством проблем на пути к порталу, вам предстоит узнать "что за пушка стреляет в ваша тушка" чтобы понять, можно ли обуздать силу этой пули и прокатиться на ней верхом, или же она просто убъет вас. Также на пути будут вас встречать молниеностные пилы, которые лишат вас жизни при первом же попадании, и кроме всего этого в нашей игре вас ждет нинзя, который потрулирует территорию и способен наносить вам сокрушительный удар и отправлять вас в начало уровня. Все уровни составлены с опорой на сложность, каждый уровень несет в себе историю и будет вести игрока по сюжетной линии, где в конца игрока ждет последний портал в счастливый конец. Удачи на платформах!

\subsubsection{Ход игры и сюжет}
На первом уровне нас встречает единсвтенная платформа и огромная пропасть, которую игроку перестоит перелететь на пуле от турели, придется подобрать правильный угол, чтобы пуля смогла доставить игрока прямо в портал.

\begin{figure}[H] 
    \centering
    \includegraphics[width=0.5\textwidth]{First.png} 
    \caption{Первый уровень.}
    \label{fig:player_character}
\end{figure}

Итак мы долетели на пуле и оказались на следующем уровне, где нас встречает патрульный и две пилы, прийдется заучить тайминги и время ходьбы нашего патрульного, который охраняет единственную платформу, которая нужна нам чтобы перейти на новый уровень.

\begin{figure}[H] 
    \centering
    \includegraphics[width=0.5\textwidth]{Second.png} 
    \caption{Второй уровень.}
    \label{fig:player_character}
\end{figure}

Не успели мы спуститься с небес на землю, как нас отправляют на арбиту земли, где внизу виднеются облака а наверху уже космос. Здесь нам стоит отточить навыки прыжков и не попасться на летящие вверх и вниз пули.

\begin{figure}[H] 
    \centering
    \includegraphics[width=0.5\textwidth]{Third.png} 
    \caption{Третий уровень.}
    \label{fig:player_character}
\end{figure}

И снова очередной портал забрасывает нас в неочевидные места нашей планеты, уже в зимний биом, где здешние турели выпускают не дружелюбные пули, которые будут доставлять нас до портала, а попытаются нас убить. И конечно же не оставят нас в покое и летающие по всей карте пилы. 

\begin{figure}[H] 
    \centering
    \includegraphics[width=0.5\textwidth]{Fourth.png} 
    \caption{Четвертый уровень.}
    \label{fig:player_character}
\end{figure}

И наконец финальный уровень, на который нас отправит последний портал на далекую другую планету даже не в солнечной системе, так как мы можем наюлюдать синии биомы. И так на этом уровне вас не ждем какой-либо босс или еще чего-то, боссы тут разработчики, который расчитали нужные тайминги пил и сделали так, что этот уровень является самым сложным для прохождения, так как игроку нужно будет расчитать посекундно что он будет делать, чтобы обойти 2 пилы и турель, которые разместили разработчики данной игры.

\begin{figure}[H] 
    \centering
    \includegraphics[width=0.5\textwidth]{Fifth.png} 
    \caption{Пятый уровень.}
    \label{fig:player_character}
\end{figure}

\subsection{Физическая модель}
Законы физики в Урюпинске отличаются от наших.  Ядерная катастрофа, случившаяся 52 года назад, принесла в мир новое явление - аномалии, которые до сих пор не до конца изучены. 
По необъяснимым причинам часть мира теперь состоит из левитирующих земляных платформ, способных выдерживать лишь вес 1 слона и 3 слонят, благо наш герой весит гораздо меньше.
\begin{figure}[H] % Используем [H], чтобы зафиксировать место изображения
    \centering
    \includegraphics[width=0.5\textwidth]{Platform.png} % Замените "player.png" на точное имя вашего файла
    \caption{Визуализация левитирующих платформ.}
    \label{fig:player_character}
\end{figure}
Вскоре люди поняли, что это не единственная аномалия. Все военные турели обрели интеллект и ненависть к людям. Теперь если между ними и человеком нет препятствий они начинают автоматическую стрельбу по нему. Турели поднялись в небо и зависли, а снаряды, которыми они стреляют, перестали падать на землю и в некоторых случаях покидают атмосферу планеты и отправляются в бескрайний космос.
\begin{figure}[H] % Используем [H], чтобы зафиксировать место изображения
    \centering
    \includegraphics[width=0.5\textwidth]{Turret1.png} % Замените "player.png" на точное имя вашего файла
    \includegraphics[width=0.5\textwidth]{Turret2.png} % Замените "player.png" на точное имя вашего файла
    \caption{Турели, направленные на игрока.}
    \label{fig:player_character}
\end{figure}
Через 5 месяцев после катастрофы люди заметили еще оду аномалию, связанную с турелями. Даже вес человека никак не влияет на полет снарядов, из-за чего особо отважные люди стали путешествовать верхом на пулях, вдохновившись Мюнхгаузеном.
\begin{figure}[H] % Используем [H], чтобы зафиксировать место изображения
    \centering
    \includegraphics[width=0.5\textwidth]{Bullet.png} % Замените "player.png" на точное имя вашего файла
    \caption{Игрок верхом на пуле.}
    \label{fig:player_character}
\end{figure}
Путешествуя по Урюпинску вы можете наткнуться на смертоносные летающие циркулярные пилы, разрубающие людей как масло.
\begin{figure}[H] % Используем [H], чтобы зафиксировать место изображения
    \centering
    \includegraphics[width=0.5\textwidth]{Saw.png} % Замените "player.png" на точное имя вашего файла
    \caption{Пила.}
    \label{fig:player_character}
\end{figure}
К сожалению, живые существа тоже подверглись влиянию радиации. Мир наводнили опасные жабы-каратисты, убивающие людей одним касанием.

Еще одной особенностью мира стали порталы зайдя в которые, человек оказывается в случайном месте, после чего портал навсегда закрывается.
\begin{figure}[H] % Используем [H], чтобы зафиксировать место изображения
    \centering
    \includegraphics[width=0.5\textwidth]{Portal.png} % Замените "player.png" на точное имя вашего файла
    \caption{Портал.}
    \label{fig:player_character}
\end{figure}
В нашей игре главный герой безоружен и не может дать отпора врагам, поэтому вам придется тщательно обдумывать, как избегать столкновения с противниками.

Управление осуществляется кнопками AD + SPACE – влево, вправо и прыжок. При длительном зажатии кнопки прыжка его высота увеличивается.

\subsection{Персонаж игрока}
\begin{itemize}
    \item Имя персонажа: \\
        Рыцарь Лидарь (Lidar Knight)
    \item Роль в игре: \\
        Игровой персонаж (протагонист,  аватар игрока)
    \item Описание в игровом мире:\\
        Лидарь — это воплощение доблести, самоотверженности и стратегического мышления. Его рыцарский кодекс остался неизменным, даже несмотря на механическую оболочку. Он сдержан, редко проявляет эмоции, но обладает глубоким чувством справедливости. Постепенно, взаимодействуя с жителями нового мира, он начинает восстанавливать свою человечность, учась заново чувствовать, дружить и верить в лучшее.

        Так Рыцарь Лидарь становится связующим звеном между прошлым, настоящим и будущим, сохраняя баланс между технологиями и человеческими ценностями.
        
        Рыцарь Лидарь — это высокотехнологичный воин из футуристического мира, где технологии и средневековые традиции слились в уникальную экосистему. Оснащенный экзоскелетом и встроенными сенсорами лидарного типа, персонаж сочетает в себе силу и точность. Лидарная технология, встроенная в его броню, позволяет ему сканировать окружение, видеть невидимые объекты и реагировать на скрытые угрозы, что делает его незаменимым в опасных миссиях.
    \item Внешний вид:
        \begin{itemize}
            \item Экзоскелет, покрытый металлическими пластинами, отражает брутальный стиль средневековых доспехов.
            \item Голова скрыта под шлемом с сенсорами, напоминающим герб рыцаря.
            \item Гибкость конструкции позволяет выполнять сложные акробатические движения.
            \item Детали костюма ярко освещаются красными огнями во время активации способностей, добавляя персонажу визуального акцента и подчеркивая его технологическую природу.
        \end{itemize}
    \item История персонажа:\\
        Рыцарь Лидарь, настоящее имя которого утеряно во времени, был создан как часть секретного проекта в далеком будущем, где человечество столкнулось с угрозой полного уничтожения. Мир захватили странные пространственные аномалии, создающие зоны хаоса, где законы физики и логики нарушались. Эти аномалии разрастались, угрожая поглотить все живое.
        Для борьбы с угрозой был разработан "Проект Лидар" — инициатива, объединяющая древние принципы рыцарской доблести с передовыми технологиями. Лидарь стал первым и единственным успешным результатом эксперимента, сочетающим человеческий разум с лидарной системой — сложнейшей технологией, позволяющей воспринимать окружающий мир на молекулярном уровне. Его броня была изготовлена из редчайших материалов, способных выдерживать искажения пространства, а разум — усилен искусственным интеллектом, помогающим анализировать данные и принимать мгновенные решения.
    
\end{itemize}

\begin{figure}[H] 
    \centering
    \includegraphics[width=0.5\textwidth]{player.png} 
    \caption{Аватар игрока - Рыцарь Лидарь.}
    \label{fig:player_character}
\end{figure}

\subsection{Элементы игры}
Этот раздел составляет описание всех элементов, которые встречаются в игре. Так, например, в этом разделе самое место подробно описать следующие игровые элементы (в зависимости от жанра):

\begin{itemize}
    \item Юниты и строения — RTS, TBS;
    \item Предметы (items) — RTS, RPG;
    \item Оружие — FPS, RPG;
    \item NPC и персонажи;
    \begin{figure}[H] 
    \centering
    \includegraphics[width=0.5\textwidth]{Idle.png} 
    \caption{Враг - патрульный.}
    \label{fig:player_character}
    \end{figure}
    \item Транспортные средства — simulators;
    \item Карты;
    \item Другое.
\end{itemize}

При описании этих элементов укажите их назначение (смысл), влияние на игровой мир и игрока, параметры и особенности и вообще всё, что существенно для этих элементов.

\subsection{Искусственный интеллект}
Данный пункт находится в разработке, и будет представлен как применение AI for NPC




\end{document}
