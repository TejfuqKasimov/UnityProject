\documentclass[a4paper,12pt]{article}
\usepackage{graphicx}
\usepackage{hyperref}
\usepackage{geometry}
\usepackage[utf8]{inputenc}  % Для кодировки UTF-8
\usepackage[russian]{babel}   % Для поддержки русского языка
\geometry{top=1in, bottom=1in, left=1in, right=1in}

\title{Дизайн-документ 2D Платформера}
\date{\today}

\begin{document}

\maketitle

\section{Введение}
Данный документ представляет собой дизайн-документ для 2D платформера. Его цель — предоставить полное описание концепции, игровых механик, технических особенностей, графического и звукового оформления, а также других аспектов разработки. Для эффективного использования данного документа ознакомьтесь с нижеприведенной информацией.

\subsection{Организация содержимого документа}
Документ разделен на тематические разделы, охватывающие ключевые аспекты разработки игры. Основные главы включают:
\begin{itemize}
    \item Введение
    \item Концепция
    \item Будем вносить остальную информацию по ходу создания дизайн-документа
\end{itemize}
Каждый раздел имеет чёткую структуру и содержит ссылки на связанные разделы для удобства навигации.

\subsection{Ссылки и используемые материалы}
Для подготовки документа были использованы следующие материалы:
\begin{itemize}
    \item Открытые библиотеки игровых шрифтов, ассетов и референсов, соответствующих лицензиям Creative Commons.
    \item Документация используемых движков и фреймворков, включая \textit{Unity}.
\end{itemize}
Все материалы защищены соответствующими авторскими правами и используются в рамках установленных лицензий.
\subsection{История изменений}
\begin{itemize}
    \item Версия 1.0 (17.11.2024) — начальная версия документа, созданы главы: Введение, Концепция
\end{itemize}

\subsection{Список авторов}
Документ подготовлен следующими авторами:
\begin{itemize}
    \item Орешкин Максим
    \item Выставкин Константин
    \item Касимов Тейфук
    \item Царапкин Егор
    \item Яновский Дмитрий
    \item Смыков Виктор
\end{itemize}

\subsection{Условные обозначения, сокращения и соглашения}
\begin{itemize}
    \item \textbf{2D} — двумерный формат графики.
\end{itemize}

\subsection{Прочие сведения}
Для чтения и понимания документа желательно иметь общее представление о разработке игр, включая основные концепции геймдизайна, программирования и художественного оформления. При необходимости, разделы документа содержат пояснения для менее знакомых терминов и понятий.

\section{Концепция}
\subsection{Введение}
В 2D-шутере игрок управляет рыцарем с загадочным именем «Лидарь», уворачиваясь от летящих дисков бензопилы и беспощадной турели «Чаппи», плюющейся своим огненным дождём из пуль, чтобы достичь портала и перейти на следующую локацию.
Динамичный геймплей рассчитаны на любителей быстрых реакций, поэтому стоит думать и, конечно же, не стоит медлить, мой друг. Строго 6+.

\subsection{Жанр и аудитория}
Сведения о жанре и целевой аудитории:

• Жанр
    
    2D-шутер

• Возрастная группа
    
    Дети 6+, обладающие наименьшей степенью реакций и взрослые: любители аркад и шутеров (от третьего лица), а также игроки, которые хотят прокачать быстроту реакции.

• Другие сведения о позиционировании игры
    
    Благодаря простоте управления и одновременно высокой сложности прохождения последних уровней игра понравится как любителям, так и новичкам.

\subsection{Основные особенности игры}
\begin{itemize}
    \item Механика игры отличается своей простотой.
    \item Особенность подобранных фонов в игре может порадовать игрока при прохождении.
    \item При прохождении игроку нужна бдительность своих действий и ловкость рук, что делает её весьма интересной.
    \item Необычное сочетание вражеского оружия (диски бензопилы и пули).
    \item Перемещение между уровнями оформлено в виде телепорта.
    \item Постепенное увеличение сложности.
    \item Среднее прохождение игры 12-17 минут
\end{itemize}

\subsection{Описание игры}
Игрок управляет рыцарем Лидарём, который должен пройти серию из 5-ти уровней, избегая попадания вражеских снарядов по нему. Враги атакуют Лидаря с помощью вращающихся кровавых дисков бензопилы и пуль, испускаемых турелями.
Главная цель игрока на каждом уровне — добраться до портала, чтобы перейти на следующий. Сложность игры возрастает с каждым новым уровнем, увеличивая количество и скорость вражеских снарядов.

\subsection{Предпосылки создания}
Новаторский 2D-шутер с использованием градиентной стилизованной графики поможет расширить спрос на рынке аркадных игр.
Простая механика и несложный геймплей, но важна заинтересованность и вовлечённость пользователя - он вырабатывается из-за нарастающей сложности игры. 
Лицензирование не требуется.

\subsection{Платформа}
Перечислите платформы, на которых планируется создание игры. Для РС-платформы укажите минимальные и рекомендуемые системные требования. Если игра требует дополнительного оборудования (например, модем), укажите это.  
\begin{table}[h!]
\centering
\begin{tabular}{|c|c|c|}
\hline

Требования & \textbf{Минимальные} & \textbf{Рекомендуемые} \\ \hline
Операционная система & Windows 7 & Windows 10\\ \hline
Процессор& 2 ядра, 2.0 ГГц & 4 ядра, 3.0 ГГц\\ \hline
ОЗУ& 4 ГБ& 8 ГБ\\ \hline
CD-ROM привод & - & - \\ \hline
Свободное место на HDD&200 МБ & 1ГБ\\ \hline
Видеокарта & Графическое ядро & 1 ГБ видеопамяти \\ \hline
Звуковая карта& - & - \\ \hline
Управление& Клавиатура, мышь & Клавиатура, мышь\\ \hline
\end{tabular}

\caption{Системные требования}
\label{tab:example_table}
\end{table}

\end{document}
