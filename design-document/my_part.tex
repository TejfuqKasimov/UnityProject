\documentclass[a4paper,12pt]{article}
\usepackage{graphicx}
\usepackage{hyperref}
\usepackage{geometry}
\usepackage[utf8]{inputenc}  % Для кодировки UTF-8
\usepackage[russian]{babel}   % Для поддержки русского языка
\geometry{top=1in, bottom=1in, left=1in, right=1in}

\title{Дизайн-документ 2D Платформера}
\date{\today}

\begin{document}

\maketitle

\section{Введение}
Данный документ представляет собой дизайн-документ для 2D платформера. Его цель — предоставить полное описание концепции, игровых механик, технических особенностей, графического и звукового оформления, а также других аспектов разработки. Для эффективного использования данного документа ознакомьтесь с нижеприведенной информацией.

\subsection{Организация содержимого документа}
Документ разделен на тематические разделы, охватывающие ключевые аспекты разработки игры. Основные главы включают:
\begin{itemize}
    \item Введение
    \item Концепция
    \item Будем вносить остальную информацию по ходу создания дизайн-документа
\end{itemize}
Каждый раздел имеет чёткую структуру и содержит ссылки на связанные разделы для удобства навигации.

\subsection{Ссылки и используемые материалы}
Для подготовки документа были использованы следующие материалы:
\begin{itemize}
    \item Открытые библиотеки игровых шрифтов, ассетов и референсов, соответствующих лицензиям Creative Commons.
    \item Документация используемых движков и фреймворков, включая \textit{Unity}.
\end{itemize}
Все материалы защищены соответствующими авторскими правами и используются в рамках установленных лицензий.
\end{document}
