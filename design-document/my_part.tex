\documentclass[a4paper,12pt]{article}
\usepackage{graphicx}
\usepackage{hyperref}
\usepackage{geometry}
\usepackage[utf8]{inputenc}  % Для кодировки UTF-8
\usepackage[russian]{babel}   % Для поддержки русского языка
\geometry{top=1in, bottom=1in, left=1in, right=1in}

\title{Дизайн-документ 2D Платформера}
\date{\today}

\begin{document}

\maketitle

\section{Введение}
Данный документ представляет собой дизайн-документ для 2D платформера. Его цель — предоставить полное описание концепции, игровых механик, технических особенностей, графического и звукового оформления, а также других аспектов разработки. Для эффективного использования данного документа ознакомьтесь с нижеприведенной информацией.

\subsection{Организация содержимого документа}
Документ разделен на тематические разделы, охватывающие ключевые аспекты разработки игры. Основные главы включают:
\begin{itemize}
    \item Введение
    \item Концепция
    \item Будем вносить остальную информацию по ходу создания дизайн-документа
\end{itemize}
Каждый раздел имеет чёткую структуру и содержит ссылки на связанные разделы для удобства навигации.

\subsection{Ссылки и используемые материалы}
Для подготовки документа были использованы следующие материалы:
\begin{itemize}
    \item Открытые библиотеки игровых шрифтов, ассетов и референсов, соответствующих лицензиям Creative Commons.
    \item Документация используемых движков и фреймворков, включая \textit{Unity}.
\end{itemize}
Все материалы защищены соответствующими авторскими правами и используются в рамках установленных лицензий.
\subsection{История изменений}
\begin{itemize}
    \item Версия 1.0 (17.11.2024) — начальная версия документа, созданы главы: Введение, Концепция
\end{itemize}

\subsection{Список авторов}
Документ подготовлен следующими авторами:
\begin{itemize}
    \item Орешкин Максим
    \item Выставкин Константин
    \item Касимов Тейфук
    \item Царапкин Егор
    \item Яновский Дмитрий
    \item Смыков Виктор
\end{itemize}

\subsection{Условные обозначения, сокращения и соглашения}
\begin{itemize}
    \item \textbf{2D} — двумерный формат графики.
\end{itemize}

\subsection{Прочие сведения}
Для чтения и понимания документа желательно иметь общее представление о разработке игр, включая основные концепции геймдизайна, программирования и художественного оформления. При необходимости, разделы документа содержат пояснения для менее знакомых терминов и понятий.

\section{Концепция}
\subsection{Введение}
В 2D-шутере игрок управляет рыцарем с загадочным именем «Лидарь», уворачиваясь от летящих дисков бензопилы и беспощадной турели "Чаппи", плюющейся своим огненным дождём из пуль, чтобы достичь портала и перейти на следующую локацию.
Динамичный геймплей рассчитаны на любителей быстрых реакций, поэтому стоит думать и, конечно же, не стоит медлить, мой друг. Строго 6+.

\end{document}
